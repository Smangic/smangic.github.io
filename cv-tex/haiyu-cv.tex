\documentclass[10pt, letterpaper]{article}

% Packages:
\usepackage[
    ignoreheadfoot, % set margins without considering header and footer
    top=1.4 cm, % seperation between body and page edge from the top
    bottom=1.4 cm, % seperation between body and page edge from the bottom
    left=2 cm, % seperation between body and page edge from the left
    right=2 cm, % seperation between body and page edge from the right
    footskip=1.0 cm, % seperation between body and footer
    % showframe % for debugging 
]{geometry} % for adjusting page geometry
\usepackage[explicit]{titlesec} % for customizing section titles
\usepackage{tabularx} % for making tables with fixed width columns
\usepackage{array} % tabularx requires this
\usepackage[dvipsnames]{xcolor} % for coloring text
\definecolor{primaryColor}{RGB}{0, 79, 144} % define primary color
\usepackage{enumitem} % for customizing lists
\usepackage{fontawesome5} % for using icons
% enable Chinese characters:
\usepackage{xeCJK}
\setCJKmainfont{PingFang SC}
\usepackage{amsmath} % for math
\usepackage[
    pdftitle={Haiyu's CV},
    pdfauthor={Haiyu WANG},
    pdfcreator={LaTeX with RenderCV},
    colorlinks=true,
    urlcolor=primaryColor
]{hyperref} % for links, metadata and bookmarks
\usepackage[pscoord]{eso-pic} % for floating text on the page
\usepackage{calc} % for calculating lengths
\usepackage{bookmark} % for bookmarks
\usepackage{lastpage} % for getting the total number of pages
\usepackage{changepage} % for one column entries (adjustwidth environment)
\usepackage{paracol} % for two and three column entries
\usepackage{ifthen} % for conditional statements
\usepackage{needspace} % for avoiding page brake right after the section title
\usepackage{iftex} % check if engine is pdflatex, xetex or luatex

% Ensure that generate pdf is machine readable/ATS parsable:
\ifPDFTeX
    \input{glyphtounicode}
    \pdfgentounicode=1
    \usepackage[T1]{fontenc}
    \usepackage[utf8]{inputenc}
    \usepackage{lmodern}
\fi

\usepackage[default, type1]{sourcesanspro} 

% Some settings:
\AtBeginEnvironment{adjustwidth}{\partopsep0pt} % remove space before adjustwidth environment
\pagestyle{empty} % no header or footer
\setcounter{secnumdepth}{0} % no section numbering
\setlength{\parindent}{0pt} % no indentation
\setlength{\topskip}{0pt} % no top skip
\setlength{\columnsep}{0.15cm} % set column seperation
\makeatletter
\let\ps@customFooterStyle\ps@plain % Copy the plain style to customFooterStyle
\patchcmd{\ps@customFooterStyle}{\thepage}{
    \color{gray}\textit{\small Haiyu WANG - Page \thepage{} of \pageref*{LastPage}}
}{}{} % replace number by desired string
\makeatother
\pagestyle{customFooterStyle}

\titleformat{\section}{
    % avoid page braking right after the section title
    \needspace{4\baselineskip}
    % make the font size of the section title large and color it with the primary color
    \Large\color{primaryColor}
}{
}{
}{
    % print bold title, give 0.15 cm space and draw a line of 0.8 pt thickness
    % from the end of the title to the end of the body
    \textbf{#1}\hspace{0.15cm}\titlerule[0.8pt]\hspace{-0.1cm}
}[] % section title formatting

\titlespacing{\section}{
    % left space:
    -1pt
}{
    % top space:
    0.3 cm
}{
    % bottom space:
    0.2 cm
} % section title spacing

% \renewcommand\labelitemi{$\vcenter{\hbox{\small$\bullet$}}$} % custom bullet points
\newenvironment{highlights}{
    \begin{itemize}[
        topsep=0.10 cm,
        parsep=0.10 cm,
        partopsep=0pt,
        itemsep=0pt,
        leftmargin=0.4 cm + 10pt
    ]
}{
    \end{itemize}
} % new environment for highlights

\newenvironment{highlightsforbulletentries}{
    \begin{itemize}[
        topsep=0.10 cm,
        parsep=0.10 cm,
        partopsep=0pt,
        itemsep=0pt,
        leftmargin=10pt
    ]
}{
    \end{itemize}
} % new environment for highlights for bullet entries


\newenvironment{onecolentry}{
    \begin{adjustwidth}{
        0.2 cm + 0.00001 cm
    }{
        0.2 cm + 0.00001 cm
    }
}{
    \end{adjustwidth}
} % new environment for one column entries

\newenvironment{twocolentry}[2][]{
    \onecolentry
    \def\secondColumn{#2}
    \setcolumnwidth{\fill, 4.5 cm}
    \begin{paracol}{2}
}{
    \switchcolumn \raggedleft \secondColumn
    \end{paracol}
    \endonecolentry
} % new environment for two column entries

\newenvironment{twocolentry-pub}[2][]{
    \onecolentry
    \def\secondColumn{#2}
    \setcolumnwidth{\fill, 2.5 cm}
    \begin{paracol}{2}
}{
    \switchcolumn \raggedleft \secondColumn
    \end{paracol}
    \endonecolentry
} % new environment for two column entries

\newenvironment{threecolentry}[3][]{
    \onecolentry
    \def\thirdColumn{#3}
    \setcolumnwidth{1 cm, \fill, 3.5 cm}
    \begin{paracol}{3}
    {\raggedright #2} \switchcolumn
}{
    \switchcolumn \raggedleft \thirdColumn
    \end{paracol}
    \endonecolentry
} % new environment for three column entries

\newenvironment{header}{
    \setlength{\topsep}{0pt}\par\kern\topsep\centering\color{primaryColor}\linespread{1.5}
}{
    \par\kern\topsep
} % new environment for the header

%! can be used to place text on the page foreground
% \newcommand{\placelastupdatedtext}{% \placetextbox{<horizontal pos>}{<vertical pos>}{<stuff>}
%   \AddToShipoutPictureFG*{% Add <stuff> to current page foreground
%     \put(
%         \LenToUnit{\paperwidth-2 cm-0.2 cm+0.05cm},
%         \LenToUnit{\paperheight-1.0 cm}
%     ){\vtop{{\null}\makebox[0pt][c]{
%         \small\color{gray}\textit{Last updated in September 2024}\hspace{\widthof{Last updated in September 2024}}
%     }}}%
%   }%
% }%

% save the original href command in a new command:
\let\hrefWithoutArrow\href

% new command for external links:
\renewcommand{\href}[2]{\hrefWithoutArrow{#1}{\ifthenelse{\equal{#2}{}}{ }{#2 }\raisebox{.15ex}{\footnotesize \faExternalLink*}}}


\begin{document}
    \newcommand{\AND}{\unskip
        \cleaders\copy\ANDbox\hskip\wd\ANDbox
        \ignorespaces
    }
    \newsavebox\ANDbox
    \sbox\ANDbox{}
    %! can be used to place text on the page foreground
    % \placelastupdatedtext
    \begin{header}
        \fontsize{26 pt}{26 pt}
        \textbf{Haiyu WANG (汪海玉)}

        \vspace{0.25 cm}

        \normalsize
        \mbox{{\footnotesize\faMapMarker*}\hspace*{0.13cm}Tokyo, Japan}%
        \kern 0.25 cm%
        \AND%
        \kern 0.25 cm%
        \mbox{\hrefWithoutArrow{mailto:haiyu@g.ecc.u-tokyo.ac.jp}{{\footnotesize\faEnvelope[regular]}\hspace*{0.13cm}haiyu@g.ecc.u-tokyo.ac.jp}}%
        \kern 0.25 cm%
        \AND%
        \kern 0.25 cm%
        \mbox{\hrefWithoutArrow{tel:+81-080-9704-1125}{{\footnotesize\faPhone*}\hspace*{0.13cm}080 9704 1125}}%
        \kern 0.25 cm%
        \AND%
        \kern 0.25 cm%
        \mbox{\hrefWithoutArrow{https://cv.haiyu.wang/}{{\footnotesize\faLink}\hspace*{0.13cm}https://cv.haiyu.wang}}%
        % \kern 0.25 cm%
        % \AND%
        % \kern 0.25 cm%
        % \mbox{\hrefWithoutArrow{https://linkedin.com/in/yourusername}{{\footnotesize\faLinkedinIn}\hspace*{0.13cm}yourusername}}%
        % \kern 0.25 cm%
        % \AND%
        % \kern 0.25 cm%
        % \mbox{\hrefWithoutArrow{https://github.com/yourusername}{{\footnotesize\faGithub}\hspace*{0.13cm}}}%
    \end{header}

    \vspace{0.3 cm - 0.3 cm}


    % \section{Welcome to RenderCV!}



        
    %     \begin{onecolentry}
    %         \href{https://rendercv.com}{RenderCV} is a LaTeX-based CV/resume version-control and maintenance app. It allows you to create a high-quality CV or resume as a PDF file from a YAML file, with \textbf{Markdown syntax support} and \textbf{complete control over the LaTeX code}.
    %     \end{onecolentry}

    %     \vspace{0.2 cm}

    %     \begin{onecolentry}
    %         The boilerplate content was inspired by \href{https://github.com/dnl-blkv/mcdowell-cv}{Gayle McDowell}.
    %     \end{onecolentry}


    
    % \section{Quick Guide}

    % \begin{onecolentry}
    %     \begin{highlightsforbulletentries}


    %     \item Each section title is arbitrary and each section contains a list of entries.

    %     \item There are 7 unique entry types: \textit{BulletEntry}, \textit{TextEntry}, \textit{EducationEntry}, \textit{ExperienceEntry}, \textit{NormalEntry}, \textit{PublicationEntry}, and \textit{OneLineEntry}.

    %     \item Select a section title, pick an entry type, and start writing your section!

    %     \item \href{https://docs.rendercv.com/user_guide/}{Here}, you can find a comprehensive user guide for RenderCV.


    %     \end{highlightsforbulletentries}
    % \end{onecolentry}

    \section{Education}


    \begin{threecolentry}{\textbf{M.ENG.}}{
        Oct 2024 - Aug 2026
    }
        \textbf{The University of Tokyo}, Electrical Engineering and Information Systems (On Going)
        \begin{highlights}
            \item \textbf{Research Area:} Flexible Electronics, Signal Processing, Meta-Suface
    
            
            % (\href{https://example.com}{a link to somewhere})
            % \item \textbf{Coursework:} Computer Architecture, Comparison of Learning Algorithms, Computational Theory
        \end{highlights}
        \vspace{0.15 cm}

    \end{threecolentry}
        
        \begin{threecolentry}{\textbf{B.ENG.}}{
            Sept 2020 - May 2024
        }
            \textbf{Southern University of Science and Technology}, Communication Engineering 
            \begin{highlights}

                \item \textbf{Overall GPA:} 3.68/4.0 \quad \textbf{Major GPA:} 3.84/4.0
                \item \textbf{Horors:} Excellent Graduates (10\%), Merit Student Scholarship (15\%) for 3 years
                \item \textbf{Research Area:} Signal Processing, Wireless Communication, Radio-Frequency Identification (RFID), Integrated Sensing and Communication (ISAC)
                \item \textbf{Courses:} Please refer to the \href{https://cv.haiyu.wang/uploads/Transcripts_English.pdf}{official transcript}
                
                % (\href{https://example.com}{a link to somewhere})
                % \item \textbf{Coursework:} Computer Architecture, Comparison of Learning Algorithms, Computational Theory
            \end{highlights}
        \end{threecolentry}


    
    \section{Publications}



        
        \begin{samepage}
            \begin{twocolentry-pub}{
                Jun 2024
            }
                \textbf{Welded Workpiece Image Acquisition System}

                \vspace{0.10 cm}

                \mbox{\textbf{\textit{Haiyu WANG}}}
                \vspace{0.10 cm}

        \href{http://epub.cnipa.gov.cn/cred/CN221210339U}{CN221210339U, Utility Model Patent, China}
        \vspace{0.15cm}
            \end{twocolentry-pub}

            \begin{twocolentry-pub}{
                Apr 2024
            }
                \textbf{Passive Respiration Detection via mmWave Communication Signal under Interference}

                \vspace{0.1 cm}

                \mbox{Kehan WU*}, \mbox{Renqi CHEN*}, \mbox{\textbf{\textit{Haiyu WANG}}}, \mbox{Chenqing JI}, \mbox{Jiayuan ZHU}, \mbox{Guang WU}
                \vspace{0.1cm}

        \href{https://ieeexplore.ieee.org/document/10570770}{2024 IEEE Wireless Communications and Networking Conference (WCNC)}

            \end{twocolentry-pub}

            \vspace{0.15 cm}

            \begin{twocolentry-pub}{
                Jan 2023
            }
                \textbf{Implementation of Anti-quantum
                Communication System using Software-Defined Radio}

                \vspace{0.10 cm}

                \mbox{Hongjia YANG}, \mbox{Jiarui XU}, \mbox{\textbf{\textit{Haiyu WANG}}}, \mbox{Chaofan WEN}, \mbox{Guang WU}
                \vspace{0.10 cm}

        \href{https://ieeexplore.ieee.org/document/10043576}{2023 IEEE International Conference on Consumer Electronics (ICCE)}
            \end{twocolentry-pub}


        \end{samepage}


    
    \section{Projects}



        
        \begin{twocolentry}{
            \small{Graduatation Project}
            
            Distinguished Undergraduate Thesis Award (Top 10\%)
        }
            \textbf{Non-destructive Detection of Tree Internal Internal Structure Based on RFID}
            \begin{highlights}
                % \item \small{Honored as Distinguished Undergraduate Thesis Award (10\%).}
                \item \small{Developed a rapid, low-cost, non-destructive system for detecting internal structures of trees using RFID technology, addressing a key challenge in ecological conservation.}
                \item Built a hardware prototype and automated data processing scripts, achieving detection within 1 minute.
                \item Expanded application to agarwood trees with machine learning algorithm,  and optimized internal structure detection accuracy to 94\%.
            \end{highlights}
        \end{twocolentry}


        \vspace{0.15 cm}

        \begin{twocolentry}{
            \small{C/C++ Program Design}
            
            Score: 96 \href{https://github.com/Smangic/SUSTech_CS205_2022_HCI}{[Github]}
        }
            \textbf{Gesture Recognition for Human-Computer Interaction (HCI)}
            \begin{highlights}
                \item \small{Developed a gesture recognition system using OpenCV based on convex hull detection.}
                \item Achieved efficient human-computer interaction such as cursor control and some shortcuts.
                \item Tools Used: C++, OpenCV
            \end{highlights}
        \end{twocolentry}


        \vspace{0.15 cm}

        \begin{twocolentry}{
            \small{Data Structure and Algorithm}
            
            Score: 100 [\href{https://github.com/Smangic/DSAA-B-Final-Project}{Github]}
        }
            \textbf{Simulation of Cell Life with Quadtree Structure}
            \begin{highlights}
                \item \small{Simulated the cellular life using Quadtree algorithms, balancing the addition, search, and deletion of cells. And}
                different kinds of cells are different in size, color and moving patterns, and their behaviors will be influenced by their surroundings.
                \item Optimized the simulation of 10,000 cells to run at 15fps without graphics and 3,000 cells at 15 fps with a full real-time GUI.
                \item Tools Used: JAVA
            \end{highlights}
        \end{twocolentry}

        \vspace{0.15 cm}
        \begin{twocolentry}{
            \small{Analog Circuits Laboratory}
            
            Score: 98\href{https://github.com/Smangic/Laser-keyboard}{[Github]}
        }
            \textbf{Laser Keyboard}
            \begin{highlights}
                \item \small{Designed a usable and portable laser keyboard prototype that can project a keyboard layout on any flat surface.}
                \item Utilized OpenCV to detect finger localization for interaction based on contour detection.
                \item Tools Used: Python, OpenCV
            \end{highlights}
        \end{twocolentry}

    \section{Experience}



        
        \begin{twocolentry}{
            Hong Kong, China

        Jul 2023 - Aug 2023
        }
            \textbf{City University of Hong Kong, Dept. of Biomedical Engineering}, Research Assistant
            \begin{highlights}
                \item Participated in the research of wireless gesture recognition with RFID tattoo.
            \end{highlights}
        \end{twocolentry}


        \vspace{0.2 cm}

        \begin{twocolentry}{
            Chengdu, China

        Mar 2023 - Jun 2023
        }
            \textbf{Chengdu Neton Inc.}, Remote R\&D Intern
            \begin{highlights}
                \item Research and development of automatic measurement system for penetration.
                \item Measured and recorded the penetration during laser welding process by Computer Vision (OpenCV, Python).
                \item Achieved a satisfying detection precision with test samples. The outcomes met the requirement of my employer.
                \item A related utility model patent is granted, and an invention patent is pending.
            \end{highlights}
        \end{twocolentry}

    
    \section{Technologies}



        
        \begin{onecolentry}
            \textbf{Languages:} Python(4 years), JAVA (4 years), MATLAB ( 3 years), C/C++ (2 years)
        \end{onecolentry}

        \vspace{0.2 cm}

        \begin{onecolentry}
            \textbf{Tools:} LabVIEW (2 years), ANSYS Electronics Desktop (1 year), Blender (half a year)
        \end{onecolentry}


    \section{Additional Information}
    \begin{highlights}
        \item Teaching assistant of course \textit{Fundamentals of Electric Circuits} in 2022.
        \item Invited to give review lessons on \textit{Analog Circuits}, \textit{Communication Principles}, and \textit{Engineering Mathematics}. Recordings have got over 1000+ views on my personal \href{https://space.bilibili.com/11220679}{Bilibili channel}.
    \end{highlights}


\end{document}